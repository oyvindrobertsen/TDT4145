% Short Sectioned Assignment
% LaTeX Template
% Version 1.0 (5/5/12)
%
% This template has been downloaded from:
% http://www.LaTeXTemplates.com
%
% Original author:
% Frits Wenneker (http://www.howtotex.com)
%
% License:
% CC BY-NC-SA 3.0 (http://creativecommons.org/licenses/by-nc-sa/3.0/)
%
%%%%%%%%%%%%%%%%%%%%%%%%%%%%%%%%%%%%%%%%%

%----------------------------------------------------------------------------------------
%   PACKAGES AND OTHER DOCUMENT CONFIGURATIONS
%----------------------------------------------------------------------------------------

\documentclass[paper=a4, fontsize=11pt]{scrartcl} % A4 paper and 11pt font size

\usepackage[utf8]{inputenc}
\usepackage[T1]{fontenc} % Use 8-bit encoding that has 256 glyphs
\usepackage{fourier} % Use the Adobe Utopia font for the document - comment this line to return to the LaTeX default
\usepackage[norsk]{babel} % English language/hyphenation
\usepackage{amsmath,amsfonts,amsthm} % Math packages

\usepackage{lipsum} % Used for inserting dummy 'Lorem ipsum' text into the template
\usepackage{graphicx} % Used for graphs and images

\usepackage{sectsty} % Allows customizing section commands
\allsectionsfont{\centering \normalfont\scshape} % Make all sections centered, the default font and small caps

\usepackage{fancyhdr} % Custom headers and footers
\pagestyle{fancyplain} % Makes all pages in the document conform to the custom headers and footers
\fancyhead{} % No page header - if you want one, create it in the same way as the footers below
\fancyfoot[L]{} % Empty left footer
\fancyfoot[C]{} % Empty center footer
\fancyfoot[R]{\thepage} % Page numbering for right footer
\renewcommand{\headrulewidth}{0pt} % Remove header underlines
\renewcommand{\footrulewidth}{0pt} % Remove footer underlines
\setlength{\headheight}{13.6pt} % Customize the height of the header

\numberwithin{equation}{section} % Number equations within sections (i.e. 1.1, 1.2, 2.1, 2.2 instead of 1, 2, 3, 4)
\numberwithin{figure}{section} % Number figures within sections (i.e. 1.1, 1.2, 2.1, 2.2 instead of 1, 2, 3, 4)
\numberwithin{table}{section} % Number tables within sections (i.e. 1.1, 1.2, 2.1, 2.2 instead of 1, 2, 3, 4)

\setlength\parindent{0pt} % Removes all indentation from paragraphs - comment this line for an assignment with lots of text

%----------------------------------------------------------------------------------------
%   TITLE SECTION
%----------------------------------------------------------------------------------------

\newcommand{\horrule}[1]{\rule{\linewidth}{#1}} % Create horizontal rule command with 1 argument of height

\title{%
\normalfont \normalsize
\textsc{NTNU - MTDT - TDT4145} \\ [25pt] % Your university, school and/or department name(s)
\horrule{0.5pt} \\[0.4cm] % Thin top horizontal rule
\huge {Øving 1} \\ % The assignment title
\horrule{2pt} \\[0.5cm] % Thick bottom horizontal rule
}

\author{Øyvind Robertsen} % Your name

\date{\normalsize\today} % Today's date or a custom date

\begin{document}

\maketitle % Print the title

%----------------------------------------------------------------------------------------
%   PROBLEM 1
%----------------------------------------------------------------------------------------
\section*{Task 1}

a)
\begin{tabular}{rl}
\textsc{Data:}  & Known facts that have implicit meaning that can recorded \\
                & as entries in a database. \\

\textsc{Database:} & A collection of related data. \\

\textsc{Database system:}   & A program (or collection thereof) that enables users to create \\
                & and maintain a database.
\end{tabular}

\vspace{0.5cm}

b) In a traditional file system, data would usually be distributed, instead of centralized. E.g:

In a large corporation each local branch would have files on each of their employees, containing info relevant only to that branch. The main branch would also have files on every employee in every local branch, containing possibly only payment info and other HR-related attributes. \\

c) A database system would reduce redundancies, and gather data more centralized, enabling the same ``files'' to be used across applications. Query speed would also be greatly increased.
\newpage

%----------------------------------------------------------------------------------------
%   PROBLEM 2
%----------------------------------------------------------------------------------------
\section*{Task 2}

\begin{figure}[h!]
\centering
\includegraphics[]{oppg2.pdf}
\caption{ER model for task 2}
\end{figure}

\newpage
%----------------------------------------------------------------------------------------
%   PROBLEM 3
%----------------------------------------------------------------------------------------
\section*{Task 3}

\begin{figure}[h!]
\centering
\includegraphics[width=0.9\textwidth]{oppg3.pdf}
\end{figure}

\newpage
%----------------------------------------------------------------------------------------
%   PROBLEM 3
%----------------------------------------------------------------------------------------
\section*{Task 4}

\begin{figure}[h!]
\centering
\includegraphics[width=0.9\textwidth]{oppg4.pdf}
\end{figure}
\end{document}
