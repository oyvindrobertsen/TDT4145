\documentclass[a4paper, 12pt] {article}

\include{setup}

\begin{document}
\pagenumbering{Roman}
% Upper part of the page. The '~' is needed because \\
% only works if a paragraph has started.
\begin{titlepage}
\begin{center}
\includegraphics[width=0.15\textwidth]{img/ntnulogo.PNG}~\\[1cm]

\textsc{\LARGE NTNU}\\[1.5cm]

\textsc{\Large TDT4145 - Datamodellering og databasesystemer}\\[0.5cm]

% Title
\HRule \\[0.4cm]
{ \huge \bfseries Øving 3}\\[0.5cm]
{\large \textit{Mathias Ose og \O yvind Robertsen}}\\[0.2cm]
\HRule \\[1.5cm]



\vfill

% Bottom of the page
{\large \today}
\end{center}
\end{titlepage}

\newpage

\tableofcontents
\newpage

\pagenumbering{arabic}

\section{Oppgave 1}

\subsection{a)}

\textbf{Nøkkel:} Den enkelte eller den minimale mengden identifiserende attributtene ved en tabell. Må være unikt identifiserende for hver rad i tabellen. Kan være en enkelt attributt eller en mengde attributter.


\textbf{Supernøkkel:} En mengde attributter i en tabell som sammen fungerer som nøkkel for alle rader i tabellen. Kan også defineres som en mengde av attributter i en relasjon der alle andre attributter er funksjonelt avhengige av supernøkkelattributtene.


\textbf{Funksjonell avhengighet:} En funksjonell avhengighet $X \rightarrow Y$, der $X$ og $Y$ er mengder av attributter, betyr at verdiene av attributtene i $X$ bestemmer verdiene av $Y$.

\subsection{b)}

Tillukningen ($X^+$) til en mengde attributter $X$ er mengden av alle attributter i relasjonen R som er funksjonelt avhengig av $X$.

Algoritme for $X^+$, med hensyn på en mengde funksjonelle avhengigheter F:

\begin{lstlisting}[language=Python, caption=Tillukningsalgoritme, mathescape]
$X^+$ = $X$
$X^+_{gammel}$ = null
while not $X^+_{gammel} == X^+$:
    $X^+_{gammel}$ = $X^+$
    for functional dependency $Y \rightarrow Z$ in F:
        if $Y$ in $X^+$:
            $X^+$ = $X^+ \cup Z$
\end{lstlisting}

\subsection{c)}

\begin{gather*}
    a^+ = \{a, e\} \\
    ab^+ = \{a, b, c, d, e\} \\
    e^+ = \{e\}
\end{gather*}

\subsection{d)}

For at en mengde attributter skal være en supernøkkel, må tillukningen av mengden inneholde alle attributter i relasjonen. For en mengde attributter $X$ fra en relasjon $R$ avgjør vi om $X$ er en supernøkkel ved å finne $X^+$ og sjekke om $X^+$ inneholder alle attributter i $R$.

For at en supernøkkel også skal være en nøkkel, må den være \textit{minimal}. Dvs. at vi kan finne ut om en supernøkkel $SK$ også er en nøkkel ved å fjerne en og en attributt og sjekke om hver restmengde er en supernøkkel.

\subsection{e)}

For at projeksjonene $R_1$ og $R_2$ skal ha tapsløs-join-egenskapen må det gå an å joine $R_1$ og $R_2$ sammen og få $R$. Formelt har vi at $R_1$ og $R_2$ har tapsløs-join-egenskapen hvis og bare hvis

\begin{gather*}
(R_1 \cap R_2 \rightarrow R_1 - R_2 ) \in F^+ \vee (R_1 \cap R_2 \rightarrow R_2 - R_1) \in F^+
\end{gather*}

Det er naturlig å anta at det kun er mulig å ha tapsløs-join dersom $R_1 \cap R_2 \rightarrow R_1 \vee R_2$. Funksjonelle avhengigheter i R kan gi oss den formelle definisjonen over, der snittet mellom $R_1$ og $R_2$ ikke nødvendigvis er $R_1$ eller $R_2$. Dette på grunn av at funksjonelle avhengigheter kan gi oss muligheten til å manipulere snittet slik at det blir lik enten den ene eller den andre.

\subsection{f)}

Vi har relasjonen $R(a,b,c,d)$ med $F = \{b \rightarrow c\}$. Dette gir oss:

\begin{enumerate}
\item $R_1(a,b,c)$ og $R_2(b,c,d)$
    \begin{itemize}
        \item $R_1 \cap R_2 \rightarrow \{b,c\}$
        \item $R_1 - R_2 \rightarrow \{a\} \notin F^+$
        \item $R_2 - R_2 \rightarrow \{d\} \notin F^+$
        \item Ikke tapsløs join
    \end{itemize}
\item $R_1(a,b,d)$ og $R_2(b,c,d)$
    \begin{itemize}
        \item $R_1 \cap R_2 \rightarrow \{b,d\}$
        \item $R_1 - R_2 \rightarrow \{a\} \notin F^+$
        \item $R_2 - R-1 \rightarrow \{c\} \in F^+$
        \item Tapsløs join
    \end{itemize}
\item $R_1(a,b,d)$ og $R_2(b,c)$
    \begin{itemize}
    \item $R_1 \cap R_2 \rightarrow \{b\}$
    \item $R_1 - R_2 \rightarrow \{a,d\} \notin F^+$
    \item $R_2 - R_1 \rightarrow \{c\} \in F^+$
    \item Tapsløs join
    \end{itemize}
\end{enumerate}

\subsection{g)}

For at en relasjon $R$ skal være på tredje normalform, må følgende krav oppfylles:

\begin{itemize}
\item R er på andre normalform
\item Alle attributter som ikke er en del av en nøkkel, er direkte avhengig av hele nøkkelen. Dvs. at transitiv avhengighet ikke er lov
\end{itemize}

\subsection{h)}

Vi har en relasjon $R(A,B,C,D)$ med funksjonelle avhengigheter $F = \{A \rightarrow B, C \rightarrow D\}$ som ikke er på 3NF. For å dekomponere til relasjoner som oppfyller kravene for 3NF, gjør vi følgende:

\begin{enumerate}
\item Finner et såkalt canonical cover $F_C$ for $F$. (I dette tilfellet er $F_C = F$)
\item For hver FD $X \rightarrow A \in F_C$, lager vi en relasjon $XA$ med $X$ som nøkkel.
    \begin{itemize}
    \item For oss blir dette $R_1(A,B)$ med $A$ som nøkkel og $R_2(C,D)$ med $C$ som nøkkel.
    \end{itemize}
\item De resterende attributtene og nøkkel samler vi i en siste relasjon. I vårt tilfelle inneholder denne relasjonen kun nøkkelen fra den opprinnelige relasjonen, nemlig $R_3(A,C)$.
\end{enumerate}

Ser at denne fremgangsmåten er fulgt for å komme fra til dekomponeringen i oppgaven. Ettersom denne fremgangsmåten bevarer funksjonelle avhengigheter og garanterer lossless-join, er dette et godt valg.

\section{Oppgave 2}
\subsection{a)}

Siden løvnodene er indeksposter med pekere til blokkidentifikatorer har dette B-treet bare indeksposter. Med 4 bytes per primærnøkkel og 4 bytes per blokkidentifikator vil hver post trenge 8 bytes. (Antar ingen ekstra meta-/administrasjonsdata) Antar vi en fyllfaktor på 67\% vil vi i gjennomsnitt få plass til $^{4096}/_{8} * 0.67 = 343$ poster i hver blokk. På nederste nivå vil vi da trenge $^{200000}/_{343} = 584$ blokker. For de tilsvarende 584 indekspostene på nivået over trenger vi to blokker. Vi trenger så en rotblokk for å indekserer disse to.

\subsection{b)}

Vi trenger en diskaksess for rotblokka, og en til for en av de to blokkene på neste nivå. Deretter må løvnodeblokka leses inn, oppdateres og skrives tilbake. (Tre aksesser her) Til sist må det samme gjøres for datalagringsblokka. (Tre til her) Til sammen får vi åtte aksesser. Dette under antagelsen at ingen av blokkene ligger i buffer.

\newpage
\section{Oppgave 3}
~\\
Setter inn verdien $12$: $12 \mod 5 = 2$\\
\begin{tabular}{|l|l|l|}
    \hline
    Adresse & Post 1 & Post 2 \\
    0       & (ledig)& (ledig)\\
    1       & (ledig)& (ledig)\\
    2       & 12     & (ledig)\\
    3       & (ledig)& (ledig)\\
    4       & (ledig)& (ledig)\\ \hline
\end{tabular}

~\\
Setter inn verdien $9$: $9 \mod 5 = 4$\\
\begin{tabular}{|l|l|l|}
    \hline
    Adresse & Post 1 & Post 2 \\
    0       & (ledig)& (ledig)\\
    1       & (ledig)& (ledig)\\
    2       & 12     & (ledig)\\
    3       & (ledig)& (ledig)\\
    4       & 9      & (ledig)\\ \hline
\end{tabular}

~\\
Setter inn verdien $3$: $3 \mod 5 = 3$\\
\begin{tabular}{|l|l|l|}
    \hline
    Adresse & Post 1 & Post 2 \\
    0       & (ledig)& (ledig)\\
    1       & (ledig)& (ledig)\\
    2       & 12     & (ledig)\\
    3       & 3      & (ledig)\\
    4       & 9      & (ledig)\\ \hline
\end{tabular}

~\\
Setter inn verdien $18$: $18 \mod 5 = 3$\\
Her må altså post 2 tas i bruk.\\
\begin{tabular}{|l|l|l|}
    \hline
    Adresse & Post 1 & Post 2 \\
    0       & (ledig)& (ledig)\\
    1       & (ledig)& (ledig)\\
    2       & 12     & (ledig)\\
    3       & 3      & 18      \\
    4       & 9      & (ledig)\\ \hline
\end{tabular}

~\\
Setter inn verdien $22$: $22 \mod 5 = 2$\\
\begin{tabular}{|l|l|l|}
    \hline
    Adresse & Post 1 & Post 2 \\
    0       & (ledig)& (ledig)\\
    1       & (ledig)& (ledig)\\
    2       & 12     & 22     \\
    3       & 3      & 18     \\
    4       & 9      & (ledig)\\ \hline
\end{tabular}

~\\
Setter inn verdien $1$: $1 \mod 5 = 1$\\
\begin{tabular}{|l|l|l|}
    \hline
    Adresse & Post 1 & Post 2 \\
    0       & (ledig)& (ledig)\\
    1       & 1      & (ledig)\\
    2       & 12     & 22     \\
    3       & 3      & 18     \\
    4       & 9      & (ledig)\\ \hline
\end{tabular}

~\\
Setter inn verdien $37$: $37 \mod 5 = 2$\\
Det finnes ingen ledig post for verdien, så vi oppretter en overflyts-blokk med to nye poster.\\
\begin{tabular}{|l|l|l|l|l|}
    \hline
    Adresse & Post 1 & Post 2 & Of1 p1 & Of1 p2 \\
    0       & (ledig)& (ledig)& ---   & ---   \\
    1       & 1      & (ledig)& ---   & ---   \\
    2       & 12     & 22     & 37     & (ledig)\\
    3       & 3      & 18     & ---   & ---   \\
    4       & 9      & (ledig)& ---   & ---   \\ \hline
\end{tabular}

~\\
Setter inn verdien $11$: $11 \mod 5 = 1$\\
\begin{tabular}{|l|l|l|l|l|}
    \hline
    Adresse & Post 1 & Post 2 & Of1 p1 & Of1 p2 \\
    0       & (ledig)& (ledig)& ---   & ---   \\
    1       & 1      & 11     & ---   & ---   \\
    2       & 12     & 22     & 37     & (ledig)\\
    3       & 3      & 18     & ---   & ---   \\
    4       & 9      & (ledig)& ---   & ---   \\ \hline
\end{tabular}

~\\
Setter inn verdien $87$: $87 \mod 5 = 2$\\
\begin{tabular}{|l|l|l|l|l|}
    \hline
    Adresse & Post 1 & Post 2 & Of1 p1 & Of1 p2 \\
    0       & (ledig)& (ledig)& ---   & ---   \\
    1       & 1      & 11     & ---   & ---   \\
    2       & 12     & 22     & 37     & 87     \\
    3       & 3      & 18     & ---   & ---   \\
    4       & 9      & (ledig)& ---   & ---   \\ \hline
\end{tabular}

~\\
Setter inn verdien $2$: $2 \mod 5 = 2$\\
Det finnes ingen ledig post for verdien, så vi må opprette en ny overflyts-blokk.\\
\begin{tabular}{|l|l|l|l|l|l|l|}
    \hline
    Adresse & Post 1 & Post 2 & Of1 p1 & Of1 p2 & Of2 p1 & Of2 p2 \\
    0       & (ledig)& (ledig)& ---   & ---   & ---   & ---   \\
    1       & 1      & 11     & ---   & ---   & ---   & ---   \\
    2       & 12     & 22     & 37     & 87     & 2      & (ledig)\\
    3       & 3      & 18     & ---   & ---   & ---   & ---   \\
    4       & 9      & (ledig)& ---   & ---   & ---   & ---   \\ \hline
\end{tabular}

~\\
Setter inn verdien $1$: $1 \mod 5 = 1$\\
Det finnes ingen ledig post for verdien, så vi må opprette en ny overflyts-blokk.\\
\begin{tabular}{|l|l|l|l|l|l|l|}
    \hline
    Adresse & Post 1 & Post 2 & Of1 p1 & Of1 p2 & Of2 p1 & Of2 p2 \\
    0       & (ledig)& (ledig)& ---   & ---   & ---   & ---   \\
    1       & 1      & 11     & 1      & (ledig)& ---   & ---   \\
    2       & 12     & 22     & 37     & 87     & 2      & (ledig)\\
    3       & 3      & 18     & ---   & ---   & ---   & ---   \\
    4       & 9      & (ledig)& ---   & ---   & ---   & ---   \\ \hline
\end{tabular}


\end{document}
