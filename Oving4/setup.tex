\usepackage[margin=1in]{geometry}
\usepackage[framemethod=TikZ]{mdframed}
\usepackage{graphicx}
\usepackage{xcolor,colortbl}
\usepackage{listings} % Allows code-listings
\usepackage{wrapfig}
\usepackage{courier} % proper monospace font for code
\usepackage{lscape}
\usepackage[utf8]{inputenc}
\usepackage[norsk]{babel}
\usepackage[T1]{fontenc}
\usepackage{rotating}
\usepackage{epstopdf}
\usepackage{hyperref} %linking fra content  table
\usepackage{xstring} %string-manipulation
\usepackage{enumitem} %used for enumerate-manipulation
\usepackage{float} %used to properly place float-objects (figures)
\usepackage{titlesec, blindtext, color}
\usepackage{tabularx,ragged2e,booktabs,caption}
\usepackage{ulem} %used for strikeout
\usepackage{verbatim} %used for block-commenting
\usepackage[toc,page]{appendix}
\usepackage{url}
\usepackage{amsmath}

\definecolor{gray75}{gray}{0.75}
\definecolor{gray1}{gray}{0.97}
\definecolor{gray2}{gray}{0.90}
\definecolor{gray3}{gray}{0.80}
\definecolor{gray4}{gray}{0.63}

\lstset{
    basicstyle=\footnotesize\ttfamily, % Standardschrift
    %numbers=left,               % Ort der Zeilennummern
    numberstyle=\tiny,          % Stil der Zeilennummern
    %stepnumber=2,               % Abstand zwischen den Zeilennummern
    numbersep=5pt,              % Abstand der Nummern zum Text
    tabsize=2,                  % Groesse von Tabs
    extendedchars=\true,         %
    breaklines=true,            % Zeilen werden Umgebrochen
    keywordstyle=\color{red},
    frame=b,         
 %  keywordstyle=[1]\textbf,    % Stil der Keywords
 %  keywordstyle=[2]\textbf,    %
 %  keywordstyle=[3]\textbf,    %
 %  keywordstyle=[4]\textbf,   \sqrt{\sqrt{}} %
    % stringstyle=\color{white}\ttfamily, % Farbe der String
    showspaces=false,           % Leerzeichen anzeigen ?
    showtabs=false,       % Tabs anzeigen ?
    inputencoding=utf8,
    xleftmargin=17pt,
    framexleftmargin=17pt,
    framexrightmargin=5pt,
    framexbottommargin=4pt,
    %backgroundcolor=\color{lightgray},
    showstringspaces=false      % Leerzeichen in Strings anzeigen ? 
}


%\DeclareCaptionFont{blue}{\color{blue}} 

%\captionsetup[lstlisting]{singlelinecheck=false, labelfont={blue}, textfont={blue}}
\usepackage{caption}
\DeclareCaptionFont{white}{\color{white}}
\DeclareCaptionFormat{listing}{\colorbox[cmyk]{0.43, 0.35, 0.35,0.01}{\parbox{\textwidth}{\hspace{15pt}#1#2#3}}}
\captionsetup[lstlisting]{format=listing,labelfont=white,textfont=white, singlelinecheck=false, margin=0pt, font={bf,footnotesize}}


\newcommand{\hsp}{\hspace{20pt}}
\newcommand{\HRule}{\rule{\linewidth}{0.5mm}}



\newcommand{\img}[2]{ %images with borders
	\begin{figure}[H]
	\centering
	\fcolorbox{black}{black}{\includegraphics[width=14cm]{img/#1}}
	\caption[#2]{#2}
	\end{figure}
}

\newcommand{\imgC}[3]{ %images with borders and custom parameters
	\begin{figure}[H]
	\centering
	\fcolorbox{black}{black}{\includegraphics[#2]{img/#1}}
	\caption[#3]{#3}
	\end{figure}
}


\newcommand{\liste}[1]{\begin{enumerate}[label=\textbf{#1:\arabic*}]}
\newcommand{\inn}{\begin{enumerate}[label*=\textbf{.\arabic*}]}
\newcommand{\ut}{\end{enumerate}}


\newcommand{\refer}[5]{\bibitem {#1} #2. "#3" \textit{#4} (#5).}

%\newcommand{\labs}{\label{sec:utdypning}}
%\newcommand{\labl}[1]{\label{itm:#1}}
%\newcommand{\refs}{(se side~\pageref{sec:utdypning})}
%\newcommand{\refl}[1]{\ref{itm:#1} - }


%\parindent=5pt
%\baselineskip=0pt
%\parskip=0pt

\hypersetup{ %Brukes for � fjerne farger fra linker i content table
    colorlinks,
    citecolor=black,
    filecolor=black,
    linkcolor=black,
    urlcolor=black
}

\titleformat{\chapter}[hang]{\Huge\bfseries}{\thechapter\hsp\textcolor{gray75}{|}\hsp}{0pt}{\Huge\bfseries}


